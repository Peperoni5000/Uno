\documentclass{article}

\usepackage{graphicx}
\usepackage[utf8]{inputenc}
\usepackage{amsmath}
\usepackage[spanish]{babel}

\title{Modelos Matemáticos Discretos}
\author{JP}
\begin{document}
\maketitle
\section{Ecuaciones en diferencias}
\subsection{Primer orden}

Sabemos que $$\lim_{x\to\infty}\frac{1}{x}=0$$

Calcular los valores propios de $$A=
\begin{pmatrix}
1 & 2 \\
\pi & 4
\end{pmatrix}
$$

\huge
\textbf{PARA TODO EPSILON MAYOR QUE CERO EXISTE UN DELTA MAYOR QUE CERO TAL QUE $$|F(X)-F(A)|<EPSILON SIEMPRE QUE |X-A|< DELTA$$}



Consideremos e valor de una inversion de \$1000 que acumula interés de 1\% cada mes

EL valor de la inversión cuando han transcurrido $n$ meses es $$x_n=1000(1.01)^n$$

Una grafica del resultado es:

\begin{center}
\includegraphics[width=8cm]{grafica}
\end{center}

Para encontrar este resultado, ocupamos que
$$\sum_{i=0}^{n-1}a^i=\frac{1-a^n}{1-a}$$

\begin{center}
\begin{tabular}{|c|r|}
\hline
Mes & Valor \\
\hline 
0 & 1000 \\
\hline 
1 & 1010 \\
\hline 
2 & 1020.1 \\
\hline 
3 & 1030.301 \\
\hline 
\end{tabular}

\huge
\textbf{NIÑOS, NO HAGAN ESTO EN CASA}
\end{center}

\subsection{Segundo orden}

\end{document}
